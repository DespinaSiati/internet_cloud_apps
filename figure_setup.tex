\usepackage[sfdefault]{roboto} % a bit lighter than Times--no
% osf in math
\usepackage[T1]{fontenc} % best for Western European languages
\usepackage{textcomp} % required to get special symbol
\usepackage[utf8]{inputenc}
\usepackage[varqu,varl,scaled=.98]{inconsolata}% a typewriter font must
% be defined
\usepackage{amsmath,amsthm}
\usepackage[libertine,cmintegrals,bigdelims,vvarbb]{newtxmath}
\usepackage[scr=rsfso]{mathalfa}
\usepackage{bm}% load after all math to give access to bold math
%After loading math package, switch to osf in text.
%\useosf % for osf in normal text

\usepackage[usenames,dvipsname]{xcolor,colortbl}

\usepackage{makeidx}

\usepackage{amsmath}
\usepackage{amssymb}

\usepackage{calc}

\usepackage{mathtools}
\DeclarePairedDelimiter{\ceil}{\lceil}{\rceil}
\DeclarePairedDelimiter{\floor}{\lfloor}{\rfloor}

\usepackage{multirow}

\usepackage{tikz}
\usetikzlibrary{shapes}
\usetikzlibrary{shapes.multipart}
\usetikzlibrary{calc}
\usetikzlibrary{positioning}
\usetikzlibrary{patterns}
\usetikzlibrary{chains}
\usetikzlibrary{arrows}
\usetikzlibrary{arrows.meta}
\usetikzlibrary{decorations.pathmorphing}
\usetikzlibrary{decorations.pathreplacing}
\usetikzlibrary{matrix}
\usetikzlibrary{datavisualization.formats.functions}
\usetikzlibrary{math}

\usepackage{pgfplots}
%% this only produces an error message, although pgfplots.pdf
%% says that it is valid:
\pgfplotsset{compat=1.15}